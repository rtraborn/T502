\documentclass[12pt]{article}
\usepackage{hyperref}
\textwidth=7in
\textheight=9.5in
\topmargin=-1in
\headheight=0in
\headsep=.5in
\hoffset  -.85in

\pagestyle{empty}

\renewcommand{\thefootnote}{\fnsymbol{footnote}}
\begin{document}

\begin{center}
{\bf BIOT-T502 \textbf{Biotechnology Laboratory}  \\ Tues/Thurs 4 - 6:15 PM,  Room: Jordan Hall A302
}
\end{center}

\setlength{\unitlength}{1in}

\begin{picture}(6,.1) 
\put(0,0) {\line(1,0){6.25}}
\end{picture}

\renewcommand{\arraystretch}{2}

\vskip.25in
\noindent\textbf{Instructor:} R. Taylor Raborn, PhD
\vskip.1in
\noindent\textbf{Office:} Simon Hall 205B
\vskip.1in
\noindent\textbf{Email:} rtraborn@indiana.edu
\vskip.1in
\noindent\textbf{Office Hours:} By appointment.\footnotemark
\vskip.25in

\noindent\textbf{Textbook:} None required. Electronic copies of selected readings will be posted on Canvas as necessary.\\

\vspace*{.15in}

\footnotetext{Please arrange a time with me either in person or via email. I ask this because I may be in the lab or in a seminar, and may not be in my office if you stop by unannounced.}

\vspace*{.15in}
\noindent \textbf{Objectives}: 

This four-week module will cover next-generation sequencing (NGS) and an introduction to computational analysis of gene expression data. The laboratory portion will comprise RNA extraction from experimental biological samples from the nematode \textit{Pristionchus pacificus} followed by analysis of total RNA integrity using the Agilient TapeStation.  We will then submit our samples for RNA-seq library preparation  followed by NGS sequencing on the NextSeq 75 platform to IU's sequencing core at the CGB. After this phase is complete, we will spend the middle portion of the course module learning the basics of NGS analysis, including how to connect to high-performance computing (HPC) resources, now to use the Linux command line and how to edit (and write) bioinformatics workflow using a text editor prior to submitting to the job scheduler. Finally, upon obtaining the sequenced RNA-seq libraries, we will perform a differential gene expression (DGE) analysis on our samples. To conclude, students will be asked to prepare a short report on their findings. Overall, this module is intended to give students a window into NGS sequencing and one of its many applications.
\\
\vskip.25in
\noindent\textbf{Grading:} Report: 70\%; Class Attendance and Participation: 30\%

\vskip.25in

\noindent\textbf{Attendance Policy}: This course will be largely lab-based and interactive, and wherever possible I will encourage active participation so that it is clear the class is engaging with the material. Because of this, attendance and positive participation in the lab and seminars will count toward a small part (30\%) of the student's final grade for this module.

\vskip.25in
\end{document}
